\documentclass[10pt,a4paper]{moderncv}

% TODO
% - Handle empty logo cases

\moderncvstyle{banking}
\moderncvcolor{blue}
\usepackage{fontawesome}


\usepackage{nag}
\usepackage[top=1.1cm, bottom=1.1cm, left=2cm, right=2cm]{geometry}
%\usepackage[scale=0.8,top=2cm]{geometry}
%\setlength{\hintscolumnwidth}{2.5cm}

\newif\iffr
\ifthenelse{\equal{\detokenize{fr}}{\jobname}}{\frtrue}{\frfalse}
\newcommand{\lang}[2]{\iffr{#2}\else{#1}\fi}

\usepackage[english, french]{babel}
\usepackage{booktabs}
\usepackage{tabularx}
\usepackage{pbox}
\usepackage{amssymb}
\usepackage{keycommand}
\usepackage{biblatex}
\usepackage{graphicx}
\usepackage{pgffor}
\graphicspath{{logo/}}

\addbibresource{citations.bib}

\newcommand{\tpt}{\href{http://www.telecom-paristech.fr/\lang{eng/}{}}{Telecom ParisTech}}
\newcommand{\tptfull}{\tpt, Paris, France}
\newcommand{\iie}{\href{http://www.ensiie.fr}{ENSIIE}}
\newcommand{\iiefull}{\iie, Evry, France}
\newcommand{\upmc}{UPMC / \href{http://www.lip6.fr}{LIP6}}
\newcommand{\lincs}{\href{http://www.lincs.fr}{LINCS}}
\newcommand{\lincsfull}{\lincs, Paris, France}

\newcommand{\jl}{Jeremie \textsc{Leguay}}
\newcommand{\jr}{Jean-Louis \textsc{Rougier}}
\newcommand{\vc}{Vania \textsc{Conan}}
\newcommand{\paolo}{Paolo \textsc{Medagliani}}

\newcommand{\cvlink}[1]{\href{#1}{#1}}
\newcommand{\flink}[1]{\footnote{\cvlink{#1}}}
\newcommand{\slink}[1]{\textcolor{gray}{\small\cvlink{#1}}}

\newcommand{\sqb}{\textcolor{color1}{\tiny$\blacksquare$}} 

%%%%%%%%%%
% Colors %
%%%%%%%%%%

\usepackage{xcolor}

\definecolor{white}{RGB}{255,255,255}

\definecolor{darkgray}{HTML}{333333}
\definecolor{gray}{HTML}{4D4D4D}
\definecolor{lightgray}{HTML}{999999}

\definecolor{green}{HTML}{C2E15F}
\definecolor{orange}{HTML}{FDA333}
\definecolor{purple}{HTML}{D3A4F9}
\definecolor{red}{HTML}{FB4485}
\definecolor{blue}{HTML}{6CE0F1}
\definecolor{tic}{HTML}{CCCC66}
\definecolor{tac}{HTML}{666699}
\colorlet{fillheader}{gray}
\colorlet{header}{white}
\colorlet{textcolor}{gray}
\colorlet{headercolor}{gray}

%%%%%%%%%
% Fonts %
%%%%%%%%%

\usepackage{fontspec}
%\usepackage[math-style=TeX,vargreek-shape=unicode]{unicode-math}

%\newfontfamily\bodyfont[]{Helvetica Neue}
%\newfontfamily\thinfont[]{Helvetica Neue UltraLight}
%\newfontfamily\headingfont[]{Helvetica Neue Condensed Bold}

\defaultfontfeatures{Mapping=tex-text}
%\setmainfont[Mapping=tex-text, Color=textcolor]{Helvetica Neue Light}
%\setmainfont[Mapping=tex-text]{Helvetica Neue Light}

%\setmathfont{XITS Math}

%%%%%%%%%%%%%%%%
% Bibliography %
%%%%%%%%%%%%%%%%
    
%%%%%%%%%%%%%%%%%%%%%%%%%%%%%%%%%%%%%%%%%%%%%%%%%%%%%%%%%%%%%%%%%%%%%%%%%%%%%%%%

\firstname{Rémy}
\familyname{Léone}
\title{
    \lang{Research engineer, PhD student}{Ingénieur de recherche, PhD}
}
\email{remy.leone@gmail.com}
\homepage{sieben.fr}
\extrainfo{
  \faTwitter~remyleone \textbullet{} 
    \faLinkedin~remyleone  \textbullet{}
    \faGithub~sieben  \textbullet{}
    \lang{French, 26 yo}{Français, 26 ans}}

%-------------------------------------------------------------------------------

\newcommand*{\multiinclude}[1]{%
  \foreach \logo in {#1} {%
     \hspace{1mm}
     \includegraphics[height=4mm]{\logo}%
  }%
}

\newkeycommand{\experience}[year,lab,position,description,advisor,logo][0]
{
    \edef\logos{\commandkey{logo}}
    \cventry{\expandafter\multiinclude\expandafter{\logos}}
        {\commandkey{position}}
        {\commandkey{lab}}
        {\commandkey{year}}{}
        {\emph{\commandkey{description} (\lang{advisor}{encadrement}: \commandkey{advisor})}}
}

\newkeycommand{\formation}[year,lab,position,description,logo][0]
{
    \edef\logos{\commandkey{logo}}
    \cventry{\expandafter\multiinclude\expandafter{\logos}}
        {\commandkey{position}}
        {\commandkey{lab}}
        {\commandkey{year}}{}
        {\emph{\commandkey{description}}}
}

\newkeycommand{\project}[name,url,description,funding,year,comments,logo][0]
{
    {\footnotesize
        \begin{minipage}[t]{0.15\linewidth}
            \commandkey{year}
        \end{minipage}
        \begin{minipage}[t]{0.35\linewidth}
            \commandkey{logo} \commandkey{name} (\commandkey{funding})  \\
            \textcolor{lightgray}{\cvlink{\commandkey{url}}}
        \end{minipage}
        \begin{minipage}[t]{0.03\linewidth}
        \end{minipage}
        \begin{minipage}[t]{0.47\linewidth}
            {\itshape\commandkey{description}} \\
            \commandkey{comments}
        \end{minipage}
    }
    \vspace{2mm}
}

\newcommand{\projectheader}{
    {\footnotesize\bfseries
        \begin{minipage}[t]{0.15\linewidth}
          \lang{Year}{Participation}
        \end{minipage}
        \begin{minipage}[t]{0.35\linewidth}
            \lang{Name and funding source}{Nom et source de financement}
        \end{minipage}
        \begin{minipage}[t]{0.03\linewidth}
        \end{minipage}
        \begin{minipage}[t]{0.47\linewidth}
            \lang{Description and comments}{Description et commentaires}
        \end{minipage}
    }
}

%%%%%%%%%%%%%%%%%%%%%%%%%%%%%%%%%%%%%%%%%%%%%%%%%%%%%%%%%%%%%%%%%%%%%%%%%%%%%%%%

\begin{document}

\maketitle

%-------------------------------------------------------------------------------

\section{\lang{Research experience}{Expérience professionnelle}}

\experience[
  year = {Jan 2013 - $\sim$ Jan 2016},
  position = {\lang{PhD Candidate}{Doctorat}},
  lab = {Telecom Paris Tech \& Thales Communication and Security},
  advisor = {\jr, \vc},
  description = {\lang{
  Title: Smart gateways for low-power and lossy networks.
  \begin{itemize}
    \item Workflow for WSN reproducible research with an open source prototype \url{http://github.com/sieben/makesense}
    \item Worked on gateway caching proxy servers 
    \item Active \& passive monitoring of constrained networks
    \item Contiki (C), Java, Wireshark, numpy, matplotlib, pandas, IPython
  \end{itemize}
  }
  {
  Titre: Passerelle intelligente pour réseau contraint.
  \begin{itemize}
    \item Automatisation des processus de simulation et d'expériences reproductibles pour réseau
    contraints. Prototype libre disponible \url{http://github.com/sieben/makesense}{http://github.com/sieben/makesense}
    \item Conception et implémentation d'un proxy intelligent couplé à un cache
    \item Supervision active \& passive d'un réseau contraint
    \item Contiki (C), Java, Wireshark, numpy, matplotlib, pandas, IPython
  \end{itemize}
  }}
]

\experience[
  year = {2012},
  position = {\lang{Final year's internship}{Stage de fin d'études}},
  lab = {Thales Communications \& Security - Colombes},
  description = {\lang{
  Design and implementation of a proxy server for low power and lossy networks with an adaptive cache policy that match lifetime constraints with acceptable network traffic}
  {Conception et implémentation d'un serveur proxy pour réseau contraint. Cache optimal pour un objectif de durée de vie du réseau fixé} \\
  Contiki (C), Java, CouchDB
  },
  advisor = {\jl, \paolo}
]

\experience[
    year = {2010},
    lab = {CEA - DAM - Bruyères-le-Châtel},
    position = {\lang{Intern}{Stagiaire}},
    description = {
      \lang{Re-factored and debug of a Fortran nuclear physics application}{Re-factorisation et amélioration d'une application de physique nucléaire en Fortran} \\ gfortran, Sun OS, Red Hat Enterprise Linux, Shell
    },
    advisor = {Thierry Granier}
]

%-------------------------------------------------------------------------------

\section{\lang{Education}{Formation}}

%\formation[
%    year        = {2004--2008},
%    lab         = {\tptfull~\lang{and}{et} \ftfull (CIFRE)},
%    position    = {\lang{PhD thesis: ``Flow-level performance guarantees for
%        traffic in Flow-Aware Networking.''}{Thèse de doctorat: ``Garanties de
%        performance pour les flots IP dans l'architecture Flow-Aware
%        Networking.''}},
%    description = {\lang{%
%        \emph{Date}: defended on November 27th, 2014 (very honourable mention) \\
%        \emph{Directors}: M. \dk~\& M. \jr \\
%        \newline
%        \emph{Members of the jury}:\newline
%        {\small
%        \begin{tabularx}{\textwidth}{XllX}
%        Mr. \tc   & \footnotesize{Professor}                       & Telecom SudParis                     & president \\
%        Ms. \igl  & \footnotesize{Professor}                       & ENS Lyon, LIP                        & rapporteur \\
%        Mr. \su   & \footnotesize{Professor}                       & Queen Mary University of London, UK\hspace{3mm}  & rapporteur \\
%        Mr. \fm   & \footnotesize{Research Engineer}               & Alcatel Lucent Bell Labs France      & examiner \\
%        Mr. \dk   & \footnotesize{Professor}                       & Telecom ParisTech                    & academic director \\
%        Mr. \jr   & \footnotesize{Senior Researcher}\hspace{3mm}   & IRT SystemX                          & industrial director \\
%        \end{tabularx} 
%        } 
%        \newline
%        $^*$ \footnotesize{work related to the PhD had been completed, but the
%        defense had been delayed due to the different activities done
%        afterwards.}
%    }{
%        \emph{Date}: soutenue le 27 Novembre 2014 (mention Très Honorable)\\
%        \emph{Directeurs de thèse}: M. \dk~\& M. \jr \\
%        \newline
%        \emph{Membres du jury}:\newline
%        {\small
%        \begin{tabularx}{\textwidth}{XllX}
%        M. \tc     & \footnotesize{Professeur}                   & Telecom SudParis                     & président du jury \\
%        Mme. \igl  & \footnotesize{Professeur}                   & ENS Lyon, LIP                        & rapporteur \\
%        M. \su     & \footnotesize{Professeur}                   & Queen Mary University of London, UK\hspace{3mm}  & rapporteur \\
%        M. \fm     & \footnotesize{Ing. Recherche}               & Alcatel Lucent Bell Labs France      & examinateur \\
%        M. \dk     & \footnotesize{Professeur}                   & Telecom ParisTech                    & directeur de thèse académique \\
%        M. \jr     & \footnotesize{Chercheur senior}\hspace{3mm} & IRT SystemX                          & directeur de thèse industriel \\
%        \end{tabularx} 
%        } 
%        \newline
%        $^*$ \footnotesize{le travail lié à la thèse était terminé, mais la
%        soutenance a été retardée en raison des différentes activités menées
%        ensuite.}
%    }},
%    %logo        = {tpt,orange},
%]

\formation[
    year        = {2010 - 2012},
    lab         = {\iiefull},
    position    = {\lang{Degree in Computer Science engineering}{Diplôme d'Ingénieur en Informatique}},
    description = {\lang{%
        \emph{major in Operating Systems and Network Administration, Routing and
        Quality of service, Software certification and System Security.}
        }{%
        \emph{spécialités: Systèmes d'exploitation et Administration de
        réseaux; Routage et Qualité de Service; Preuve formelle et sécurité des
        systèmes.}
        }},
    %logo        = {ensiie},
]

\formation[
    year        = {2008 - 2009},
    lab         = {Lycée Fénelon, Paris, France},
    position    = {\lang{Preparatory school (French CPGE)}{Classes Préparatoires aux Grandes Ecoles}},
    description = {\lang{major in Mathematics and Physics.}{spécialités Math-Physique (MP)}},
]

%-------------------------------------------------------------------------------

\section{\lang{Knowledge and skills}{Connaissances}}

\cvitem{\lang{Theory}{Théorie}}
    {\lang{Data Analysis, Database Design, Statistics, Optimization, Numerical Analysis; language and compilation
    theory; graphs and operational research 
    }{Analyse de données, Conception de base de données, statistiques, optimisation, Analyse numérique; théorie des langages
    et de la compilation; graphes et recherche opérationnelle;}
    }

\cvitem{Administration}
    {\lang{GNU/Linux: System and network administration, Computer Architecture, Operating System} {GNU/Linux: Administration système et réseau, Architecture des ordinateurs, Systèmes d'exploitation}}

\cvitem{Web}{
  flask, django, nginx, CouchDB, PostgreSQL, redis
}

\cvitem{SCM}{
  SVN, git, CVS
}

\cvitem{\lang{Scientific software}{Logiciels scientifiques}}{
  pandas, numpy, matplotlib, ipython, \LaTeX{}, R, Octave ($\sim$ Matlab)
}

\cvitem{\lang{Virtualization}{Virtualisation}}{
  VirtualBox, Docker, Vagrant
}

\cvitem{\lang{Networking}{Réseau}}
    {\lang{Routing and Quality of Service, network architecture and
    protocols, Wireless}{Routage et Qualité de Service, architecture réseau et
    protocoles, Sans-fil}, TCP/IP, VPN, Wireshark
    }

\cvitem{\lang{Programming}{Programmation}}
    {C, Java, ASM (Intel, 68k), Python, OCaml, Shell, 
    \lang{Concurrent Programming, several contributions to free software}{Programmation concurrente, plusieurs
    contributions au logiciel libre}}

\section{\lang{Teaching}{Enseignement}}

\begin{tabularx}{\textwidth}{p{15mm}p{15mm}p{30mm}p{30mm}p{15mm}X}
\toprule
Type & \lang{Hours}{Heures} & Audience & \lang{Level}{Niveau} & \lang{Language}{Langue} & Description \\
\toprule

\lang{Tut/Lab}{TD/TP} & 60.5h & $\sim$ 30 \lang{students}{étudiants} & \lang{1st year eng.}{1A Ingénieur} & FR & \href{http://perso.telecom-paristech.fr/\~chaudet/res101/}{RES101} (BCI Réseaux) \\

    \lang{Tut/Lab}{TD/TP} & 18h & $\sim$ 10 \lang{students}{étudiants} & \lang{International M2}{M2 international} & EN & 
    \href{http://perso.telecom-paristech.fr/\~chaudet/res841/}{RES841} (Computer Networks) \\

    \lang{Tut/Lab}{TD/TP} & 21h & 26 \lang{students}{étudiants} & \lang{1st year eng.}{1A Ingénieur} & FR & 
    \href{http://www.infres.enst.fr/~bellot/java/}{INF103} (Java programming) \\

\bottomrule
\multicolumn{6}{l}{TOTAL: \textbf{99.5 \lang{h}{HETD}}}
\end{tabularx}

\newpage

\nocite{*}
\printbibliography[title={Publications}]

% {\tiny

% \begin{refsection}
%   \nocite{*}
%   \printbibliography[sorting=chronological, title={\lang{International peer-reviewed journals}{Articles dans des revues
%   d'audience internationale avec comité de rédaction}}, keyword={journal},
%   heading=subbibliography, env=my]
% \end{refsection}
% \begin{refsection}
%   \nocite{*}
%   \printbibliography[sorting=chronological, title={\lang{International peer-reviewed conferences}{Publications dans des manifestations d'audience internationale avec comité de sélection}}, keyword={conference}, heading=subbibliography]
% \end{refsection}
% \begin{refsection}
%   \nocite{*}
%   \printbibliography[sorting=chronological, title={Workshops}, keyword={workshop}, heading=subbibliography]
% \end{refsection}
% \begin{refsection}
%   \nocite{*}
%   \printbibliography[sorting=chronological, title={\lang{Selected talks}{Sélection de présentations}}, keyword={invitedtalk}, heading=subbibliography]
% \end{refsection}
% \begin{refsection}
%   \nocite{*}
%   \printbibliography[sorting=chronological, title={\lang{Selected tutorials}{Sélection de tutoriaux et TP}}, keyword={tutorial}, heading=subbibliography]
% \end{refsection}

% }

%-------------------------------------------------------------------------------


%-------------------------------------------------------------------------------

\section{\lang{Projects and collaborations}{Projets et collaborations}}

\subsection{\lang{Ongoing projects}{Projets en cours}}

\projectheader

\project[
    year        = {\lang{since}{depuis} 2012},
    name        = {IRIS},
    url         = {http://www.anr-iris.fr},
    funding     = {ANR},
    description = {All IP Networks for the Future Internet of Smart Objects},
]

\subsection{\lang{Former projects}{Projets terminés}}

\project[
    year        = {2011--2013},
    name        = {Calipso},
    url         = {http://www.ict-calipso.eu/},
    funding     = {EU/FP7},
    description = {Connect All IP-based Smart Objects},
]

%-------------------------------------------------------------------------------

\section{\lang{Community}{Communauté}}

\begin{itemize}
  \item \lang{Multiple contributions on code quality and continuous integration in Contiki}{Multiples
    contributions pour l'amélioration de la qualité du code par l'intégration continue.}
  \item \lang{Support on open source mailing list}{Assistance et aide aux débutants sur les listes
    de diffusion}
    \item \lang{IT administration in a science association}{Administrateur système pour une association de vulgarisation scientifique} (\href{http://paris-montagne.fr}{Paris Montagne})
\end{itemize}

\section{
    \lang{Additional information}{Informations diverses}
}

\cvitem{
    \lang{Languages}{Langues}
}{
    \lang{French (native)}{Français (natif)},
    \lang{English (fluent)}{Anglais (courant)},
    \lang{Spanish (high-school level)}{Espagnol (scolaire)},
}

\end{document}

